%%%%%%%%%%%%%%%%%%%%%%%%%%%%%%%%%%%%%%%%%%%%%%%%%%%%%%%%%%%%%%%%%%%%%%%%
%%%%%%%%%%%%%%%%%%%%%% Simple LaTeX CV Template %%%%%%%%%%%%%%%%%%%%%%%%
%%%%%%%%%%%%%%%%%%%%%%%%%%%%%%%%%%%%%%%%%%%%%%%%%%%%%%%%%%%%%%%%%%%%%%%%

%%%%%%%%%%%%%%%%%%%%%%%%%%%% Document Setup %%%%%%%%%%%%%%%%%%%%%%%%%%%%

% Don't like 10pt? Try 11pt or 12pt
\documentclass[10pt]{article}

% The automated optical recognition software used to digitize resume
% information works best with fonts that do not have serifs. This
% command uses a sans serif font throughout. Uncomment both lines (or at
% least the second) to restore a Roman font (i.e., a font with serifs).
%\usepackage{times}
%\renewcommand{\familydefault}{\sfdefault}

% This is a helpful package that puts math inside length specifications
\usepackage{calc}

% Layout: Puts the section titles on left side of page
\reversemarginpar

%
%         PAPER SIZE, PAGE NUMBER, AND DOCUMENT LAYOUT NOTES:
%
% The next \usepackage line changes the layout for CV style section
% headings as marginal notes. It also sets up the paper size as either
% letter or A4. By default, letter was used. If A4 paper is desired,
% comment out the letterpaper lines and uncomment the a4paper lines.
%
% As you can see, the margin widths and section title widths can be
% easily adjusted.
%
% ALSO: Notice that the includefoot option can be commented OUT in order
% to put the PAGE NUMBER *IN* the bottom margin. This will make the
% effective text area larger.
%
% IF YOU WISH TO REMOVE THE ``of LASTPAGE'' next to each page number,
% see the note about the +LP and -LP lines below. Comment out the +LP
% and uncomment the -LP.
%
% IF YOU WISH TO REMOVE PAGE NUMBERS, be sure that the includefoot line
% is uncommented and ALSO uncomment the \pagestyle{empty} a few lines
% below.
%

%% Use these lines for letter-sized paper
\usepackage[paper=letterpaper,
            %includefoot, % Uncomment to put page number above margin
            marginparwidth=1.0in,     % Length of section titles
            marginparsep=.05in,       % Space between titles and text
            margin=0.5in,             % 1 inch margins
            includemp]{geometry}

%% Use these lines for A4-sized paper
%\usepackage[paper=a4paper,
%            %includefoot, % Uncomment to put page number above margin
%            marginparwidth=30.5mm,    % Length of section titles
%            marginparsep=1.5mm,       % Space between titles and text
%            margin=25mm,              % 25mm margins
%            includemp]{geometry}

%% More layout: Get rid of indenting throughout entire document
\setlength{\parindent}{0in}

\usepackage[shortlabels]{enumitem}

% Simpler bibsections for CV sections
% (thanks to natbib for inspiration)
%
% * For lists of references with hanging indents and no numbers:
%
%   \begin{bibsection}
%       \item ...
%   \end{bibsection}
%
% * For numbered lists of references (with hanging indents):
%
%   \begin{bibenum}
%       \item ...
%   \end{bibenum}
%
%   Note that bibenum numbers continuously throughout. To reset the
%   counter, use
%
%   \restartlist{bibenum}
%
%   at the place where you want the numbering to reset.

\makeatletter
\newlength{\bibhang}
\setlength{\bibhang}{1em}
\newlength{\bibsep}
 {\@listi \global\bibsep\itemsep \global\advance\bibsep by\parsep}
\newlist{bibsection}{itemize}{3}
\setlist[bibsection]{label=,leftmargin=\bibhang,%
        itemindent=-\bibhang,
        itemsep=\bibsep,parsep=\z@,partopsep=0pt,
        topsep=0pt}
\newlist{bibenum}{enumerate}{3}
\setlist[bibenum]{label=[\arabic*],resume,leftmargin={\bibhang+\widthof{[999]}},%
        itemindent=-\bibhang,
        itemsep=\bibsep,parsep=\z@,partopsep=0pt,
        topsep=0pt}
\let\oldendbibenum\endbibenum
\def\endbibenum{\oldendbibenum\vspace{-.6\baselineskip}}
\let\oldendbibsection\endbibsection
\def\endbibsection{\oldendbibsection\vspace{-.6\baselineskip}}
\makeatother

%% Reference the last page in the page number
%
% NOTE: comment the +LP line and uncomment the -LP line to have page
%       numbers without the ``of ##'' last page reference)
%
% NOTE: uncomment the \pagestyle{empty} line to get rid of all page
%       numbers (make sure includefoot is commented out above)
%
\usepackage{fancyhdr,lastpage}
\pagestyle{fancy}
%\pagestyle{empty}      % Uncomment this to get rid of page numbers
\fancyhf{}\renewcommand{\headrulewidth}{0pt}
\fancyfootoffset{\marginparsep+\marginparwidth}
\newlength{\footpageshift}
\setlength{\footpageshift}
          {0.5\textwidth+0.5\marginparsep+0.5\marginparwidth-2in}
%\lfoot{\hspace{\footpageshift}%
%       \parbox{4in}{\, \hfill %
%                    \arabic{page} of \protect\pageref*{LastPage} % +LP
%                    \arabic{page}                               % -LP
%                    \hfill \,}}

% Finally, give us PDF bookmarks
\usepackage{color,hyperref}
\definecolor{darkblue}{rgb}{0.0,0.0,0.3}
\hypersetup{colorlinks,breaklinks,
            linkcolor=darkblue,urlcolor=darkblue,
            anchorcolor=darkblue,citecolor=darkblue}

%%%%%%%%%%%%%%%%%%%%%%%% End Document Setup %%%%%%%%%%%%%%%%%%%%%%%%%%%%


%%%%%%%%%%%%%%%%%%%%%%%%%%% Helper Commands %%%%%%%%%%%%%%%%%%%%%%%%%%%%

%%% HEADING AT TOP OF CURRICULUM VITAE

% The title (name) with a horizontal rule under it
% (optional argument typesets an object right-justified across from name
%  as well)
%
% Usage: \makeheading{name}
%        OR
%        \makeheading[right_object]{name}
%
% Place at top of document. It should be the first thing.
% If ``right_object'' is provided in the square-braced optional
% argument, it will be right justified on the same line as ``name'' at
% the top of the CV. For example:
%
%       \makeheading[\emph{Curriculum vitae}]{Your Name}
%
% will put an emphasized ``Curriculum vitae'' at the top of the document
% as a title. Likewise, a picture could be included:
%
%   \makeheading[\includegraphics[height=1.5in]{my_picutre}]{Your Name}
%
% the picture will be flush right across from the name.
\newcommand{\makeheading}[2][]%
        {\hspace*{-\marginparsep minus \marginparwidth}%
         \begin{minipage}[t]{\textwidth+\marginparwidth+\marginparsep}%
             {\huge \bfseries #2 \hfill #1}\\
             {\normalsize \url{hartwig@psfc.mit.edu} \hfill +1 314 922 6495 (cell) \hfill +1 617 253 0025 (work) #1}\\
             {\normalsize 77 Massachusetts Ave, NW17-155, Cambridge MA 02139 \hfill \href{http://www.psfc.mit.edu/~hartwig}{\url{http://www.psfc.mit.edu/$\sim$hartwig}} #1}\\
             [-0.15\baselineskip]%
             \rule{\columnwidth}{1pt}%
         \end{minipage}}

%%% SECTION HEADINGS

% The section headings. Flush left in small caps down pseudo-margin.
%
% Usage: \section{section name}
\renewcommand{\section}[1]{\pagebreak[3]%
    \vspace{1.3\baselineskip}%
    \phantomsection\addcontentsline{toc}{section}{#1}%
    \noindent\llap{\scshape\smash{\parbox[t]{\marginparwidth}{\hyphenpenalty=10000\raggedright #1}}}%
    \vspace{-\baselineskip}\par}

%%% LISTS

% This macro alters a list by removing some of the space that follows the list
% (is used by lists below)
\newcommand*\fixendlist[1]{%
    \expandafter\let\csname preFixEndListend#1\expandafter\endcsname\csname end#1\endcsname
    \expandafter\def\csname end#1\endcsname{\csname preFixEndListend#1\endcsname\vspace{-0.6\baselineskip}}}

% These macros help ensure that items in outer-type lists do not get
% separated from the next line by a page break
% (they are used by lists below)
\let\originalItem\item
\newcommand*\fixouterlist[1]{%
    \expandafter\let\csname preFixOuterList#1\expandafter\endcsname\csname #1\endcsname
    \expandafter\def\csname #1\endcsname{\csname preFixOuterList#1\endcsname\let\oldItem\item\def\item{\pagebreak[2]\oldItem}}
    \expandafter\let\csname preFixOuterListend#1\expandafter\endcsname\csname end#1\endcsname
    \expandafter\def\csname end#1\endcsname{\let\item\oldItem\csname preFixOuterListend#1\endcsname}}
\newcommand*\fixinnerlist[1]{%
    \expandafter\let\csname preFixInnerList#1\expandafter\endcsname\csname #1\endcsname
    \expandafter\def\csname #1\endcsname{\let\oldItem\item\let\item\originalItem\csname preFixInnerList#1\endcsname}
    \expandafter\let\csname preFixInnerListend#1\expandafter\endcsname\csname end#1\endcsname
    \expandafter\def\csname end#1\endcsname{\csname preFixInnerListend#1\endcsname\let\item\oldItem}}

% An itemize-style list with lots of space between items
%
% Usage:
%   \begin{outerlist}
%       \item ...    % (or \item[] for no bullet)
%   \end{outerlist}
\newlist{outerlist}{itemize}{3}
    \setlist[outerlist]{label=\enskip\textbullet,leftmargin=*}
    \fixendlist{outerlist}
    \fixouterlist{outerlist}

% An environment IDENTICAL to outerlist that has better pre-list spacing
% when used as the first thing in a \section
%
% Usage:
%   \begin{lonelist}
%       \item ...    % (or \item[] for no bullet)
%   \end{lonelist}
\newlist{lonelist}{itemize}{3}
    \setlist[lonelist]{label=\enskip\textbullet,leftmargin=*,partopsep=0pt,topsep=0pt}
    \fixendlist{lonelist}
    \fixouterlist{lonelist}

% An itemize-style list with little space between items
%
% Usage:
%   \begin{innerlist}
%       \item ...    % (or \item[] for no bullet)
%   \end{innerlist}
\newlist{innerlist}{itemize}{3}
    \setlist[innerlist]{label=\enskip\small\textbullet,leftmargin=*,parsep=0pt,itemsep=0pt,topsep=0pt,partopsep=0pt}
    \fixinnerlist{innerlist}

% An environment IDENTICAL to innerlist that has better pre-list spacing
% when used as the first thing in a \section
%
% Usage:
%   \begin{loneinnerlist}
%       \item ...    % (or \item[] for no bullet)
%   \end{loneinnerlist}
\newlist{loneinnerlist}{itemize}{3}
    \setlist[loneinnerlist]{label=\enskip\textbullet,leftmargin=*,parsep=0pt,itemsep=0pt,topsep=0pt,partopsep=0pt}
    \fixendlist{loneinnerlist}
    \fixinnerlist{loneinnerlist}

%%% EXTRA SPACE

% To add some paragraph space between lines.
% This also tells LaTeX to preferably break a page on one of these gaps
% if there is a needed pagebreak nearby.
\newcommand{\blankline}{\quad\pagebreak[3]}
\newcommand{\halfblankline}{\quad\vspace{-0.5\baselineskip}\pagebreak[3]}

%%% FORMATTING MACROS

% Uses hyperref to link DOI
\newcommand\doilink[1]{\href{http://dx.doi.org/#1}{#1}}
\newcommand\doi[1]{doi:\doilink{#1}}

% For \url{SOME_URL}, links SOME_URL to the url SOME_URL
\providecommand*\url[1]{\href{#1}{#1}}
% Same as above, but pretty-prints SOME_URL in teletype fixed-width font
\renewcommand*\url[1]{\href{#1}{\texttt{#1}}}

% For \email{ADDRESS}, links ADDRESS to the url mailto:ADDRESS
\providecommand*\email[1]{\href{mailto:#1}{#1}}
% Same as above, but pretty-prints ADDRESS in teletype fixed-width font
%\renewcommand*\email[1]{\href{mailto:#1}{\texttt{#1}}}

%\providecommand\BibTeX{{\rm B\kern-.05em{\sc i\kern-.025em b}\kern-.08em
%    T\kern-.1667em\lower.7ex\hbox{E}\kern-.125emX}}
%\providecommand\BibTeX{{\rm B\kern-.05em{\sc i\kern-.025em b}\kern-.08em
%    \TeX}}
\providecommand\BibTeX{{B\kern-.05em{\sc i\kern-.025em b}\kern-.08em
    \TeX}}
\providecommand\Matlab{\textsc{Matlab}}

% Custom hyphenation rules for words that LaTeX has trouble with
\hyphenation{bio-mim-ic-ry bio-in-spi-ra-tion re-us-a-ble pro-vid-er}

%%%%%%%%%%%%%%%%%%%%%%%% End Helper Commands %%%%%%%%%%%%%%%%%%%%%%%%%%%

%%%%%%%%%%%%%%%%%%%%%%%%% Begin CV Document %%%%%%%%%%%%%%%%%%%%%%%%%%%%

\begin{document}
\makeheading{Zachary S. Hartwig}
\vspace{0.1cm}

%%
%% In modern CV's, it seems like ``Objective'' is frowned upon. Instead,
%% incorporate it into a well-constructed cover letter. The ``More
%% information'' can go at the end of the CV, but it should not distract
%% from the section giving references available to contact.
%%
%
% \section{Objective}
%
% Placement in an academic position (i.e., faculty, postdoctoral, or
% research scientist) that allows for advanced research in distributed
% complex adaptive systems (i.e., modeling, analysis, design, and
% verification) with a particular focus on the control of engineered
% agents (e.g., for communications, control, software, electronics, and
% sustainability) and the analysis of biological phenomena (e.g.,
% self-organization, ecological rationality)
% \begin{innerlist}
% \item More information and auxiliary documents can be found at\\\url{http://www.tedpavlic.com/facjobsearch/}
% \end{innerlist}

% \section{Research Interests}
% \textbf{Applying radiation detection and particle transport simulation
%   towards solving complex problems in nuclear science and
%   engineering}: nuclear diagnostics for plasma-wall interactions in
% magnetic fusion devices; digital data acquisition and pulse
% processing; advanced particle detectors for special nuclear material
% security; computational ion beam analysis for materials science; Monte
% Carlo particle transport development and simulation; compact
% superconducting cyclotrons for nuclear medicine and security;
% production of fundamental nuclear data; conceptual engineering designs
% of magnetic fusion devices.

\newlength{\rcollength}\setlength{\rcollength}{3.0in}%
\newlength{\spacewidth}\setlength{\spacewidth}{20pt}
\newcommand\spacechar{$|$}

\section{Research Interests}
\textbf{Advancing radiation detection and particle transport simulation \\ to solve complex problems in nuclear science and engineering}\\

\begin{tabular}[t]{@{}p{\textwidth-\rcollength-\spacewidth}@{}p{\spacewidth}@{}p{\rcollength}}%

% Address box
\parbox{\textwidth-\rcollength-\spacewidth}{%
\begin{innerlist}
\item Particle detection for nuclear security
\item Monte Carlo particle transport simulation
\item Digital data acquisition and pulse processing
\item Computational ion beam materials analysis
\end{innerlist}

}
% Shorten by removing \spacechar's
& \parbox{\spacewidth}{\centering} &

% Non-snail-mail contact information
\parbox{\rcollength}{%

\begin{innerlist}
\item Nuclear diagnostic for magnetic fusion
\item Production of fundamental nuclear data
\item Conceptual designs for magnetic fusion
\item Compact superconducting cyclotrons
\end{innerlist}
%\vspace{0.40cm}
}
\end{tabular}

\vspace{0.4cm}

\section{Education}

%\newlength{\rcollength}\setlength{\rcollength}{3.0in}%
%\newlength{\spacewidth}\setlength{\spacewidth}{20pt}
%\newcommand\spacechar{$|$}
%
\begin{tabular}[t]{@{}p{\textwidth-\rcollength-\spacewidth}@{}p{\spacewidth}@{}p{\rcollength}}%

% Address box
\parbox{\textwidth-\rcollength-\spacewidth}{%
\textbf{Ph.D. in Nuclear Science}, MIT. September 2013.

\begin{innerlist}
\item Concentration: Fusion nuclear science
\item GPA: 4.7 / 5.0
\item Thesis: \emph{An accelerator-based, in-situ diagnostic for plasma-material interactions science on the Alcator C-Mod tokamak}
\end{innerlist}

}
% Shorten by removing \spacechar's
& \parbox{\spacewidth}{\centering} &

% Non-snail-mail contact information
\parbox{\rcollength}{%
\textbf{B.A. in Physics}, \href{http://www.bu.edu}{Boston University}. May 2005.
\begin{innerlist}
\item Concentration: Experimental particle physics
\item GPA: 3.7 / 4.0
\item Degree awarded \emph{summa cum laude}
\item Dean's List all 8 semesters
\end{innerlist}
\vspace{0.40cm}
}
\end{tabular}

\vspace{0.4cm}

\section{Publications}
\begin{bibenum}
   \item Z.S. Hartwig, C.B. Haakonsen, R.T. Mumgaard, \emph{et al.} An initial study of demountable, high-temperature superconducting magnets for the Vulcan tokamak conceptual design.
     \emph{Fusion Engineering and Design}, \textbf{87} (2012) 201-214.
%     \doi{10.1016/j.fusengdes.2011.10.002}

   \item Z.S. Hartwig and D.G. Whyte. Simulated plasma facing component measurements for an \emph{in-situ} surface diagnostic on
       Alcator C-Mod. \emph{Review of Scientific Instruments}, \textbf{81} 10E106 (2010).
%      \doi{10.1063/1.3478634}

   \item Z.S. Hartwig and M. Zucchetti. Neutronics studies for a compact, high-field tokamak neutron source.
     \emph{Fusion Science and Technology}, \textbf{60} (2011) 725-729.

   \item Z.S Hartwig and Y.A. Podpaly. \emph{The Magnetic Fusion Energy Formulary}. 
       Independently published. 
       \\ Available at: \href{http://www.psfc.mit.edu/~hartwig/formulary.shtml}{http://www.psfc.mit.edu/$\sim$hartwig/formulary.shtml}

   \item A. Inglis \emph{et al.}, ``Glass-Panel Li-6 Neutron Detector'', presented at the Technologies for Homeland Security (HST), IEEE International Conference on, Waltham, MA, 2012. 
       In press. 
       \\ Available at \href{http://darkmatter.bu.edu/~ainglis/lithium\_Inglis.pdf}{http://darkmatter.bu.edu/$\sim$ainglis/lithium\_Inglis.pdf}

   \item G.M. Olynyk, Z.S. Hartwig, D.G. Whyte, \emph{et al.} Vulcan: A steady-state tokamak for reactor-relevant plasma–material interaction science.
      \emph{Fusion Engineering and Design}, \textbf{87} (2012) 224-233.

   \item H.S. Barnard, Z.S. Hartwig, G.M. Olynyk, \emph{et al.} Assessing the feasibility of a high-temperature, helium-cooled vacuum vessel and first wall for the Vulcan tokamak conceptual design.
      \emph{Fusion Engineering and Design}, \textbf{87} (2012) 248-262.

   \item D.G. Whyte, G.M. Olynyk, H.S. Barnard, \emph{et al.} Reactor similarity for plasma–material interactions in scaled-down tokamaks as the basis for the Vulcan conceptual design.
      \emph{Fusion Engineering and Design}, \textbf{87} (2012) 234-247.

  \item M. Zucchetti, B. Coppy, F. Bombarda, \emph{et al.} Compact tokamak neutron sources as a first step towards hybrid fission-fusion reactors.
     \emph{Fusion Science and Technology}. In press. 
     \\ Available at \href{http://www.psfc.mit.edu/~hartwig/public\_download/Zucchetti\_TOFE\_2012\_Final.pdf}{http://www.psfc.mit.edu/$\sim$hartwig/public\_download/TOFE\_2012\_Final.pdf}

   \item D.M. Webber, V. Tishchenko, Q. Peng, \emph{et al.} Measurement of the Positive Muon Lifetime and Determination of the Fermi Constant to Part-per-Million Precision.
     \emph{Phys. Rev. Lett}, \textbf{106} (2011) 041803.

\end{bibenum}

\section{Notable Achievements}
\begin{innerlist}
\item \textit{Recipient}, Boston University Alumni Prize for Excellence in Physics. May 2005.
\item \textit{Keynote speaker}, MIT Nuclear Science and Engineering Department (NSE) Research Expo. March 2011.
\item \textit{Recipient}, MIT International Science and Technology Initative Global Seed Fund Grant. May 2011.
\item \textit{Recipient}, MIT Plasma Science and Fusion Center Award, Science Education and Outreach. July 2012.
\item \textit{Recipient}, MIT NSE Special Award, Excellence in Science Communication and Policy. May 2012.
\item \textit{Invitated talk}, Conference on the Application of Accelerators in Research and Industry. August 2012.
\item \textit{USA Cycling National Champion}, Collegiate Track Division II Team Omnium. September 2012.
\end{innerlist}

\makeheading{Zachary S. Hartwig}
\vspace{0.1cm}

\section{Hardware Expertise}
\textbf{Detector Data Acquisition}
\begin{innerlist}
  \item CAEN S.p.A. data acquisition systems, Tektronix digital oscilloscopes
\end{innerlist}

\halfblankline

\textbf{Particle Detector Construction}
\begin{innerlist}
\item Scintillator crystals, photomultiplier tubes, silicon avalanche
  photodiodes, silicon photomultiplier, signal preamplifiers,
  soldering, basic machining, detector test platforms
\end{innerlist}

\section{Computer expertise}

\textbf{Programming Languages}
\begin{innerlist}
  \item C, C$+$$+$, Java, Python, Unix shell scripting, GNU make, Matlab, IDL, Open MPI, MPICH2
\end{innerlist}

\halfblankline

\textbf{Particle Transport and Nuclear Physics Codes}
\begin{innerlist}
  \item Geant4, MCNP5/X, DAGMC CAD-based neutronics, SRIM/TRIM, EASY, NJOY, TALYS, EMPIRE
\end{innerlist}

\halfblankline

\textbf{Data acquisition, storage and analysis}
\begin{innerlist}
  \item ROOT, MDSplus
\end{innerlist}

\halfblankline

\textbf{Computer-Aided Design (CAD) and Analysis}
\begin{innerlist}
  \item Solid Edge ST5, CUBIT Tool Suite, COMSOL Multiphysics 4
\end{innerlist}

\halfblankline

\textbf{Productivity Software}
\begin{innerlist}
  \item Windows OS, Linux OS (Fedora, RHEL, Ubuntu), Emacs, Subversion, Git, \LaTeX, LibreOffice, GIMP
\end{innerlist} 


\section{Research Experience}
\begin{innerlist}
\item \textbf{Ph.D student, MIT (2007-present)}: Designed and
  demonstrated an innovative accelerator-based materials diagnostic
  for magnetic fusion devices. The research involved creating
  start-of-the-art particle transport simulations, applying advanced
  radiation detection in a challenging environment, installing a data
  acquisition system, and creating analysis tools for digital pulse
  processing of detector data. Project involved significant team work
  with Alcator C-Mod personnel and mentoring of younger
  students.\vspace{0.2cm}

\item \textbf{Consultant, Neotron Inc. (2009-2010)}: Played a key role
  in the development of an innovative lithium-6-based detector for
  homeland security. The research involved creating high-fidelity
  models of proposed detector designs using particle transport
  simulations and working with a team to optimize cost and detection
  efficiency within the engineering constraints to achieve a final
  design.\vspace{0.2cm}
  
\item \textbf{Consultant, Cyclotron Group, MIT (2010)}: Predicted the
  deleterious effects of losing control of an accelerated particle
  beam in a proposed ultracompact superconducting cyclotron
  facility. The research involved creating and benchmarking a
  high-fidelity model of the cyclotron and calculating nuclear heating
  and radiation damage to the superconducting magnets as a result of
  various accident scenarios.\vspace{0.2cm}

\item \textbf{Research Assistant, Boston University (2004-2006)}:
  Developed a particle physics simulation for the Muon g-2 Experiment,
  which searches for physics beyond the Standard Model of particle
  physics. The simulation is presently used and developed by a
  world-wide collaboration as the leading computational design tool
  for the next generation of the experiment that will run at Fermilab
  National Accelerator Laboratory in 2016. Experimental work on
  particle detection and data acquisition performed on the M$\mu$LAN
  experiment to measure the lifetime of the muon at Paul Sherrer
  Institute, Switzerland, 2005.

\end{innerlist}

\section{Leadership Experience}
\begin{innerlist}
\item \textbf{Mediator}: Conflict resolution, MIT Resistance for Easing Friction and Stress Program. January 2010.
\item \textbf{Organizer}: U.S. fusion student advocacy trip to 30 Congressional offices in Washington DC. June 2012.
\end{innerlist}

\section{Teaching Experience}
\begin{innerlist}
\item \textbf{Private tutor}: High school physics for several Boston University Academy Students. 2005-2006. 
\item \textbf{Teaching assistant}: 22.105 : Electromagnetic Interactions (Prof. D. Whyte). Fall 2010.
\item \textbf{Teaching assistant}: 22.63: Engineering Principles for Fusion Reactors (Prof. D. Whyte). Spring 2012.
\item \textbf{UROP advisor}: Florent Sainct (2009), Jake Jurewicz (2011), Gabriel Ledoux (2012, 2013)
\item \textbf{Undergraduate thesis mentor}: Lauren Chilton, MIT Class of 2012.
\end{innerlist}

% \section{Academic Coursework (MIT)}

% \newlength{\rcollengthtwo}\setlength{\rcollengthtwo}{3.1in}%

% \begin{tabular}[t]{@{}p{\textwidth-\rcollengthtwo-\spacewidth}@{}p{\spacewidth}@{}p{\rcollengthtwo}}%

% % Address box
% \parbox{\textwidth-\rcollengthtwo-\spacewidth}{%
%   \begin{innerlist}
%   \item   3.320 Atomistic Computer Materials Modeling
%   \item  18.085 Computational Science and Engineering
%   \item  22.071 Electronics, Signals, and Measurement
%   \item  22.101 Applied Nuclear Physics
%   \item  22.105 Electromagnetic Interactions
%   \item  22.106 Neutron Interactions and Applications
%   \item  22.51 Quantum Theory of Radiation Interactions
%   \item  22.611 Introduction to Plasmas Physics I
%   \end{innerlist}
% }

% & \parbox{\spacewidth}{\centering} &

% % Non-snail-mail contact information
% \parbox{\rcollengthtwo}{%

% \begin{innerlist}
% \item  22.612 Introduction to Plasmas Physics II
% \item  22.62 Fusion Energy
% \item  22.63 Engineering Principles of Fusion Reactors
% \item  22.70 Materials for Nuclear Applications
% \item  22.811 Sustainable Energy
% \item  22.814 Nuclear Non-Proliferation
% \item  22.93 Teaching Experience in Nuclear Science
% \end{innerlist}
% }

% \end{tabular}

\end{document}

%%%%%%%%%%%%%%%%%%%%%%%%%% End CV Document %%%%%%%%%%%%%%%%%%%%%%%%%%%%%

%----------------------------------------------------------------------%
% The following is copyright and licensing information for
% redistribution of this LaTeX source code; it also includes a liability
% statement. If this source code is not being redistributed to others,
% it may be omitted. It has no effect on the function of the above code.
%----------------------------------------------------------------------%
% Copyright (c) 2007, 2008, 2009, 2010, 2011 by Theodore P. Pavlic
%
% Unless otherwise expressly stated, this work is licensed under the
% Creative Commons Attribution-Noncommercial 3.0 United States License. To
% view a copy of this license, visit
% http://creativecommons.org/licenses/by-nc/3.0/us/ or send a letter to
% Creative Commons, 171 Second Street, Suite 300, San Francisco,
% California, 94105, USA.
%
% THE SOFTWARE IS PROVIDED "AS IS", WITHOUT WARRANTY OF ANY KIND, EXPRESS
% OR IMPLIED, INCLUDING BUT NOT LIMITED TO THE WARRANTIES OF
% MERCHANTABILITY, FITNESS FOR A PARTICULAR PURPOSE AND NONINFRINGEMENT.
% IN NO EVENT SHALL THE AUTHORS OR COPYRIGHT HOLDERS BE LIABLE FOR ANY
% CLAIM, DAMAGES OR OTHER LIABILITY, WHETHER IN AN ACTION OF CONTRACT,
% TORT OR OTHERWISE, ARISING FROM, OUT OF OR IN CONNECTION WITH THE
% SOFTWARE OR THE USE OR OTHER DEALINGS IN THE SOFTWARE.
%----------------------------------------------------------------------%
