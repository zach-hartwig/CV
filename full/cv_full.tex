%%%%%%%%%%%%%%%%%%%%%%%%%%%%%%%%%%%%%%%%%%%%%%%%%%%%%%%%%%%%%%%%%%%%%%%%
%%%%%%%%%%%%%%%%%%%%%% Simple LaTeX CV Template %%%%%%%%%%%%%%%%%%%%%%%%
%%%%%%%%%%%%%%%%%%%%%%%%%%%%%%%%%%%%%%%%%%%%%%%%%%%%%%%%%%%%%%%%%%%%%%%%

%%%%%%%%%%%%%%%%%%%%%%%%%%%% Document Setup %%%%%%%%%%%%%%%%%%%%%%%%%%%%

% Don't like 10pt? Try 11pt or 12pt
\documentclass[10pt]{article}


% The automated optical recognition software used to digitize resume
% information works best with fonts that do not have serifs. This
% command uses a sans serif font throughout. Uncomment both lines (or at
% least the second) to restore a Roman font (i.e., a font with serifs).
%\usepackage{times}
%\renewcommand{\familydefault}{\sfdefault}

% This is a helpful package that puts math inside length specifications
\usepackage{calc}

% Layout: Puts the section titles on left side of page
\reversemarginpar

%
%         PAPER SIZE, PAGE NUMBER, AND DOCUMENT LAYOUT NOTES:
%
% The next \usepackage line changes the layout for CV style section
% headings as marginal notes. It also sets up the paper size as either
% letter or A4. By default, letter was used. If A4 paper is desired,
% comment out the letterpaper lines and uncomment the a4paper lines.
%
% As you can see, the margin widths and section title widths can be
% easily adjusted.
%
% ALSO: Notice that the includefoot option can be commented OUT in order
% to put the PAGE NUMBER *IN* the bottom margin. This will make the
% effective text area larger.
%
% IF YOU WISH TO REMOVE THE ``of LASTPAGE'' next to each page number,
% see the note about the +LP and -LP lines below. Comment out the +LP
% and uncomment the -LP.
%
% IF YOU WISH TO REMOVE PAGE NUMBERS, be sure that the includefoot line
% is uncommented and ALSO uncomment the \pagestyle{empty} a few lines
% below.
%

%% Use these lines for letter-sized paper
\usepackage[paper=letterpaper,
            %includefoot, % Uncomment to put page number above margin
            marginparwidth=1.0in,     % Length of section titles
            marginparsep=.05in,       % Space between titles and text
            margin=0.5in,             % 1 inch margins
            includemp]{geometry}

%% Use these lines for A4-sized paper
%\usepackage[paper=a4paper,
%            %includefoot, % Uncomment to put page number above margin
%            marginparwidth=30.5mm,    % Length of section titles
%            marginparsep=1.5mm,       % Space between titles and text
%            margin=25mm,              % 25mm margins
%            includemp]{geometry}

%% More layout: Get rid of indenting throughout entire document
\setlength{\parindent}{0in}

\usepackage[shortlabels]{enumitem}

% Simpler bibsections for CV sections
% (thanks to natbib for inspiration)
%
% * For lists of references with hanging indents and no numbers:
%
%   \begin{bibsection}
%       \item ...
%   \end{bibsection}
%
% * For numbered lists of references (with hanging indents):
%
%   \begin{bibenum}
%       \item ...
%   \end{bibenum}
%
%   Note that bibenum numbers continuously throughout. To reset the
%   counter, use
%
%   \restartlist{bibenum}
%
%   at the place where you want the numbering to reset.

\makeatletter
\newlength{\bibhang}
\setlength{\bibhang}{1em}
\newlength{\bibsep}
 {\@listi \global\bibsep\itemsep \global\advance\bibsep by\parsep}
\newlist{bibsection}{itemize}{3}
\setlist[bibsection]{label=,leftmargin=\bibhang,%
        itemindent=-\bibhang,
        itemsep=\bibsep,parsep=\z@,partopsep=0pt,
        topsep=0pt}
\newlist{bibenum}{enumerate}{3}
\setlist[bibenum]{label=[\arabic*],resume,leftmargin={\bibhang+\widthof{[999]}},%
        itemindent=-\bibhang,
        itemsep=\bibsep,parsep=\z@,partopsep=0pt,
        topsep=0pt}
\let\oldendbibenum\endbibenum
\def\endbibenum{\oldendbibenum\vspace{-.6\baselineskip}}
\let\oldendbibsection\endbibsection
\def\endbibsection{\oldendbibsection\vspace{-.6\baselineskip}}
\makeatother

%% Reference the last page in the page number
%
% NOTE: comment the +LP line and uncomment the -LP line to have page
%       numbers without the ``of ##'' last page reference)
%
% NOTE: uncomment the \pagestyle{empty} line to get rid of all page
%       numbers (make sure includefoot is commented out above)
%
\usepackage{fancyhdr,lastpage}
\pagestyle{fancy}
%\pagestyle{empty}      % Uncomment this to get rid of page numbers
\fancyhf{}\renewcommand{\headrulewidth}{0pt}
\fancyfootoffset{\marginparsep+\marginparwidth}
\newlength{\footpageshift}
\setlength{\footpageshift}
          {0.5\textwidth+0.5\marginparsep+0.5\marginparwidth-2in}
%\lfoot{\hspace{\footpageshift}%
%       \parbox{4in}{\, \hfill %
%                    \arabic{page} of \protect\pageref*{LastPage} % +LP
%                    \arabic{page}                               % -LP
%                    \hfill \,}}

% Finally, give us PDF bookmarks
\usepackage{color,hyperref}
\definecolor{darkblue}{rgb}{0.0,0.0,0.3}
\hypersetup{colorlinks, breaklinks,
            linkcolor=blue,
            urlcolor=blue,
            anchorcolor=blue,
            citecolor=blue}

%%%%%%%%%%%%%%%%%%%%%%%% End Document Setup %%%%%%%%%%%%%%%%%%%%%%%%%%%%


%%%%%%%%%%%%%%%%%%%%%%%%%%% Helper Commands %%%%%%%%%%%%%%%%%%%%%%%%%%%%

%%% HEADING AT TOP OF CURRICULUM VITAE

% The title (name) with a horizontal rule under it
% (optional argument typesets an object right-justified across from name
%  as well)
%
% Usage: \makeheading{name}
%        OR
%        \makeheading[right_object]{name}
%
% Place at top of document. It should be the first thing.
% If ``right_object'' is provided in the square-braced optional
% argument, it will be right justified on the same line as ``name'' at
% the top of the CV. For example:
%
%       \makeheading[\emph{Curriculum vitae}]{Your Name}
%
% will put an emphasized ``Curriculum vitae'' at the top of the document
% as a title. Likewise, a picture could be included:
%
%   \makeheading[\includegraphics[height=1.5in]{my_picutre}]{Your Name}
%
% the picture will be flush right across from the name.
\newcommand{\makeheading}[2][]%
        {\hspace*{-\marginparsep minus \marginparwidth}%
         \begin{minipage}[t]{\textwidth+\marginparwidth+\marginparsep}%
             {\huge \bfseries #2 \hfill #1}\\
             {\normalsize \texttt{hartwig@psfc.mit.edu} \hfill +1 314 922 6495 (cell) \hfill +1 617 253 5471 (work) #1}\\
             {\normalsize 77 Massachusetts Ave, NW17-115, Cambridge MA 02139 \hfill \href{http://www.psfc.mit.edu/~hartwig}{\url{http://www.psfc.mit.edu/$\sim$hartwig}} #1}\\
             [-0.15\baselineskip]%
             \rule{\columnwidth}{1pt}%
         \end{minipage}}

%%% SECTION HEADINGS

% The section headings. Flush left in small caps down pseudo-margin.
%
% Usage: \section{section name}
\renewcommand{\section}[1]{\pagebreak[3]%
    \vspace{1.3\baselineskip}%
    \phantomsection\addcontentsline{toc}{section}{#1}%
    \noindent\llap{\scshape\smash{\parbox[t]{\marginparwidth}{\hyphenpenalty=10000\raggedright #1}}}%
    \vspace{-\baselineskip}\par}

%%% LISTS

% This macro alters a list by removing some of the space that follows the list
% (is used by lists below)
\newcommand*\fixendlist[1]{%
    \expandafter\let\csname preFixEndListend#1\expandafter\endcsname\csname end#1\endcsname
    \expandafter\def\csname end#1\endcsname{\csname preFixEndListend#1\endcsname\vspace{-0.6\baselineskip}}}

% These macros help ensure that items in outer-type lists do not get
% separated from the next line by a page break
% (they are used by lists below)
\let\originalItem\item
\newcommand*\fixouterlist[1]{%
    \expandafter\let\csname preFixOuterList#1\expandafter\endcsname\csname #1\endcsname
    \expandafter\def\csname #1\endcsname{\csname preFixOuterList#1\endcsname\let\oldItem\item\def\item{\pagebreak[2]\oldItem}}
    \expandafter\let\csname preFixOuterListend#1\expandafter\endcsname\csname end#1\endcsname
    \expandafter\def\csname end#1\endcsname{\let\item\oldItem\csname preFixOuterListend#1\endcsname}}
\newcommand*\fixinnerlist[1]{%
    \expandafter\let\csname preFixInnerList#1\expandafter\endcsname\csname #1\endcsname
    \expandafter\def\csname #1\endcsname{\let\oldItem\item\let\item\originalItem\csname preFixInnerList#1\endcsname}
    \expandafter\let\csname preFixInnerListend#1\expandafter\endcsname\csname end#1\endcsname
    \expandafter\def\csname end#1\endcsname{\csname preFixInnerListend#1\endcsname\let\item\oldItem}}

% An itemize-style list with lots of space between items
%
% Usage:
%   \begin{outerlist}
%       \item ...    % (or \item[] for no bullet)
%   \end{outerlist}
\newlist{outerlist}{itemize}{3}
    \setlist[outerlist]{label=\enskip\textbullet,leftmargin=*}
    \fixendlist{outerlist}
    \fixouterlist{outerlist}

% An environment IDENTICAL to outerlist that has better pre-list spacing
% when used as the first thing in a \section
%
% Usage:
%   \begin{lonelist}
%       \item ...    % (or \item[] for no bullet)
%   \end{lonelist}
\newlist{lonelist}{itemize}{3}
    \setlist[lonelist]{label=\enskip\textbullet,leftmargin=*,partopsep=0pt,topsep=0pt}
    \fixendlist{lonelist}
    \fixouterlist{lonelist}

% An itemize-style list with little space between items
%
% Usage:
%   \begin{innerlist}
%       \item ...    % (or \item[] for no bullet)
%   \end{innerlist}
\newlist{innerlist}{itemize}{3}
    \setlist[innerlist]{label=\enskip\small\textbullet,leftmargin=*,parsep=0pt,itemsep=0pt,topsep=0pt,partopsep=0pt}
    \fixinnerlist{innerlist}

% An environment IDENTICAL to innerlist that has better pre-list spacing
% when used as the first thing in a \section
%
% Usage:
%   \begin{loneinnerlist}
%       \item ...    % (or \item[] for no bullet)
%   \end{loneinnerlist}
\newlist{loneinnerlist}{itemize}{3}
    \setlist[loneinnerlist]{label=\enskip\textbullet,leftmargin=*,parsep=0pt,itemsep=0pt,topsep=0pt,partopsep=0pt}
    \fixendlist{loneinnerlist}
    \fixinnerlist{loneinnerlist}

%%% EXTRA SPACE

% To add some paragraph space between lines.
% This also tells LaTeX to preferably break a page on one of these gaps
% if there is a needed pagebreak nearby.
\newcommand{\blankline}{\quad\pagebreak[3]}
\newcommand{\halfblankline}{\quad\vspace{-0.5\baselineskip}\pagebreak[3]}

%%% FORMATTING MACROS

% Uses hyperref to link DOI
\newcommand\doilink[1]{\href{http://dx.doi.org/#1}{#1}}
\newcommand\doi[1]{doi:\doilink{#1}}

% For \url{SOME_URL}, links SOME_URL to the url SOME_URL
\providecommand*\url[1]{\href{#1}{#1}}
% Same as above, but pretty-prints SOME_URL in teletype fixed-width font
\renewcommand*\url[1]{\href{#1}{\texttt{#1}}}

% For \email{ADDRESS}, links ADDRESS to the url mailto:ADDRESS
\providecommand*\email[1]{\href{mailto:#1}{#1}}
% Same as above, but pretty-prints ADDRESS in teletype fixed-width font
%\renewcommand*\email[1]{\href{mailto:#1}{\texttt{#1}}}

%\providecommand\BibTeX{{\rm B\kern-.05em{\sc i\kern-.025em b}\kern-.08em
%    T\kern-.1667em\lower.7ex\hbox{E}\kern-.125emX}}
%\providecommand\BibTeX{{\rm B\kern-.05em{\sc i\kern-.025em b}\kern-.08em
%    \TeX}}
\providecommand\BibTeX{{B\kern-.05em{\sc i\kern-.025em b}\kern-.08em
    \TeX}}
\providecommand\Matlab{\textsc{Matlab}}

% Custom hyphenation rules for words that LaTeX has trouble with
\hyphenation{bio-mim-ic-ry bio-in-spi-ra-tion re-us-a-ble pro-vid-er}

%%%%%%%%%%%%%%%%%%%%%%%% End Helper Commands %%%%%%%%%%%%%%%%%%%%%%%%%%%



%%%%%%%%%%%%%%%%%%%%%%%%% Begin CV Document %%%%%%%%%%%%%%%%%%%%%%%%%%%%

\begin{document}
\makeheading{Zachary S. Hartwig, Ph.D.}
%\vspace{0.1cm}

%%
%% In modern CV's, it seems like ``Objective'' is frowned upon. Instead,
%% incorporate it into a well-constructed cover letter. The ``More
%% information'' can go at the end of the CV, but it should not distract
%% from the section giving references available to contact.
%%
%
% \section{Objective}
%
% Placement in an academic position (i.e., faculty, postdoctoral, or
% research scientist) that allows for advanced research in distributed
% complex adaptive systems (i.e., modeling, analysis, design, and
% verification) with a particular focus on the control of engineered
% agents (e.g., for communications, control, software, electronics, and
% sustainability) and the analysis of biological phenomena (e.g.,
% self-organization, ecological rationality)
% \begin{innerlist}
% \item More information and auxiliary documents can be found at\\\url{http://www.tedpavlic.com/facjobsearch/}
% \end{innerlist}

% \section{Research Interests}
% \textbf{Applying radiation detection and particle transport simulation
%   towards solving complex problems in nuclear science and
%   engineering}: nuclear diagnostics for plasma-wall interactions in
% magnetic fusion devices; digital data acquisition and pulse
% processing; advanced particle detectors for special nuclear material
% security; computational ion beam analysis for materials science; Monte
% Carlo particle transport development and simulation; compact
% superconducting cyclotrons for nuclear medicine and security;
% production of fundamental nuclear data; conceptual engineering designs
% of magnetic fusion devices.

\newlength{\rcollength}\setlength{\rcollength}{3.0in}%
\newlength{\spacewidth}\setlength{\spacewidth}{20pt}
\newcommand\spacechar{$|$}

\section{Research Interests}
\textbf{Development and application of radiation detectors, radiation
  sources, and particle \\ transport simulation to solve complex
  problems in nuclear science and engineering}\vspace{0.1cm}

\begin{tabular}[t]{@{}p{\textwidth-\rcollength-\spacewidth}@{}p{\spacewidth}@{}p{\rcollength}}%

% Address box
\parbox{\textwidth-\rcollength-\spacewidth}{%
\begin{innerlist}
\item Radiation and particle detector development
\item Satellite-born detectors and energy sources    
\item Accelerator-based nuclear and material science
\item Fusion energy nuclear science and device design
\end{innerlist}
}
% Shorten by removing \spacechar's
& \parbox{\spacewidth}{\centering} &

% Non-snail-mail contact information
\parbox{\rcollength}{%

\begin{innerlist}
\item Monte Carlo particle transport simulations
\item Digital data acquisition and analysis systems
\item Active and passive detection in nuclear security
\item Fusion plasma-material interaction science
\end{innerlist}
}
\end{tabular}

\vspace{0.4cm}

\section{Education}

%\newlength{\rcollength}\setlength{\rcollength}{3.0in}%
%\newlength{\spacewidth}\setlength{\spacewidth}{20pt}
%\newcommand\spacechar{$|$}
%
\begin{tabular}[t]{@{}p{\textwidth-\rcollength-\spacewidth}@{}p{\spacewidth}@{}p{\rcollength}}%

% Address box
\parbox{\textwidth-\rcollength-\spacewidth}{%
\textbf{Ph.D. in Nuclear Science}, MIT. February 2014.

\begin{innerlist}
\item Concentration: Fusion nuclear science
\item GPA: 4.7 / 5.0
\item Thesis: \emph{An accelerator-based in-situ diagnostic for plasma-material interactions science on magnetic fusion devices}
\end{innerlist}

}
% Shorten by removing \spacechar's
& \parbox{\spacewidth}{\centering} &

% Non-snail-mail contact information
\parbox{\rcollength}{%
\textbf{B.A. in Physics}, Boston University. May 2005.
\begin{innerlist}
\item Concentration: Experimental particle physics
\item GPA: 3.7 / 4.0
\item Degree awarded \emph{summa cum laude}
\item Recipient of Alumni Award in Physics
\item Dean's List all 8 semesters
\end{innerlist}
}
\end{tabular}

\vspace{0.4cm}

\section{Notable Achievements}
\begin{innerlist}
\item \textit{Recipient}, U.S. Department of Energy ORISE Postdoctoral Fellowship, Jan 2015.
\item \textit{Recipient}, MIT NSE Del Favero Prize in Nuclear Science and Engineering, May 2014.
\item \textit{Fellow}, 2013 Kavli Frontiers of Science.
\item \textit{Invited speaker}, Kavli Frontiers of Science Meeting, November, 2013.
\item \textit{USA Cycling National Champion}, Collegiate Track Division II Team Omnium. September 2012.
\item \textit{Recipient}, MIT NSE Special Award, Excellence in Science Communication and Policy. May 2012.
\item \textit{Recipient}, MIT Plasma Science and Fusion Center Award, Science Education and Outreach. July 2012.
\item \textit{Recipient}, MIT International Science and Technology Initative Global Seed Fund Grant. May 2011.
\item \textit{Recipient}, Boston University Alumni Prize for Excellence in Physics. May 2005.
\end{innerlist}

\section{Research Experience}
\begin{innerlist}
\item \textbf{Postdoctoral associate/fellow, MIT (2013-present)}:
  Initiated and lead a number of diverse research efforts. Lead data
  acquisition, analysis, and computation efforts for two nuclear
  security projects (Low-dose monoenergetic gamma radiography system;
  Zero knowledge warhead verification system). Conducted on-going
  efforts to bring an ultracompact superconducting cyclotron to MIT
  for nuclear security and materials research. Proposed and lead a
  collaboration with MIT Aeronautics and Astronautics Engineering
  department on novel low-cost satellite-based particle spectrometers
  and dosimeters. Established a new accelerator science and detector
  development laboratory with collaborators. Cofounded a design group
  at MIT Plasma Science and Fusion Center to pursue a new approach to
  fusion energy with private funding and advanced
  technology. Continued development of the AIMS diagnostic for
  plasma-material interaction science on the Alcator C-Mod
  tokamak.\vspace{0.2cm}
  
\item \textbf{Advisor, Tokamak Energy U.K. (2013-2014)}: Advised a
  private company on Monte Carlo neutronics simulations for magnetic
  fusion applications. The work involved training team members on
  advanced fusion neutronics and improving existing in-house
  simulation capabilities.\vspace{0.2cm}

\item \textbf{Ph.D student, MIT (2007-2013)}: Designed and
  demonstrated an innovative accelerator-based materials diagnostic
  for magnetic fusion devices. The research involved creating advanced
  particle transport simulations, applying radiation detection in a
  challenging environment, implementing a custom digital data
  acquisition system, and creating data analysis tools.\vspace{0.2cm}

\item \textbf{Advisor, Neotron Inc. (2009-2010)}: Collaborated on
  development of an innovative lithium-6-based detector for homeland
  security. The research involved using particle transport simulations
  to optimize the design and cost of the final detector.\vspace{0.2cm}
  
\item \textbf{Advisor, Cyclotron Group, MIT (2010)}: Predicted the
  impact of nuclear heating on superconducting magnets in a proposed
  ultracompact superconducting cyclotron during various operational
  scenarios using particle transport simulations.\vspace{0.2cm}

\item \textbf{Research Assistant, Boston University (2004-2006)}:
  Developed a particle physics simulation for the Muon g-2 Experiment,
  previously at Brookhave National Lab and now at Fermi National Lab.
  The simulation is presently used as a leading computational design
  tool for the next generation of the experiment. Performed
  experimental work on particle detection and data acquisition for the
  M$\mu$LAN muon lifetime experiment at Paul Sherrer Institute,
  Switzerland, 2005.
\end{innerlist}

\newpage
\makeheading{Zachary S. Hartwig, Ph.D.}

\section{Doctoral Dissertation}
\textbf{MIT Department of Nuclear Science and Engineering Ph.D. Dissertation}
\begin{innerlist}
\item Z.S. Hartwig (2013). \textit{An In-situ Accelerator-based Diagnostic
  for Plasma-Material Interactions on Magnetic Fusion Devices.}
Doctoral Dissertation, MIT, Cambridge MA,
USA.\\ \href{http://www.psfc.mit.edu/library1/catalog/reports/2010/13rr/13rr012/13rr012_full.pdf}{Available
  for download here}
\end{innerlist}

\section{Research Publications}
\textbf{Accelerator-based In-situ Materials Surveillance (AIMS)}
\begin{innerlist}

\item Z.S. Hartwig \textit{et al.} \textit{Fuel retention measurements
  in Alcator C-Mod using Accelerator-based In situ Materials
  Surveillance.} J. Nucl. Mat. \textbf{263} (2015) 73.
  \href{http://dx.doi.org/10.1016/j.jnucmat.2014.09.056}{doi:10.1016/j.jnucmat.2014.09.056}
  \vspace{0.2cm}

\item Z.S. Hartwig \textit{et al.} \textit{An in-situ
  accelerator-based diagnostic for plasma-material interactions on
  magnetic fusion devices.} Rev. Sci. Instr. \textbf{84} (2013)
  123503.
  \href{http://dx.doi.org/10.1063/1.4832420}{doi:10.1064/1.4832420}
  \vspace{0.2cm}

\item Z.S. Hartwig and D.G. Whyte. \textit{Simulated plasma-facing
  component measurements for an in-situ surface diagnostic on Alcator
  C-Mod.} Rev. Sci. Instr. \textbf{81} (2010)
  10E106.
  \href{http://dx.doi.org/10.1063/1.3478634}{doi:10.1063/1.4832420}
\end{innerlist}

\vspace{0.5cm}

\textbf{Magnetic fusion energy design and engineering}
\begin{innerlist}
\item Z.S. Hartwig \textit{et al.} \textit{An initial study of
  demountable, high-temperature superconducting magnets for the Vulcan
  tokamak conceptual design.} Fus. Eng. Design \textbf{87} (2012) 201.
  \\\href{http://dx.doi.org/10.1016/j.fusengdes.2011.10.002}{doi:10.1016/j.fusengdes.2011.10.002}
  \vspace{0.2cm}

\item G.M. Olynyk, Z.S. Hartwig, \textit{et al.} \textit{Vulcan: a
  steady-state tokamak for reactor-relevant plasma-material
  interaction science.} Fus. Eng. Design \textbf{87} (2012) 224.
  \href{http://dx.doi.org/10.1016/j.fusengdes.2011.12.009}{doi:10.1016/j.fusengdes.2011.12.009}
  \vspace{0.2cm}

\item G.M. Olynyk, Z.S. Hartwig, \textit{et al.} \textit{Assessing the
  feasibility of a high-temperature, helium-cooled vacuum vessel and
  first wall for the Vulcan tokamak conceptual design.} Fus. Eng. Design \textbf{87} (2012) 248.
  \href{http://dx.doi.org/10.1016/j.fusengdes.2011.12.018}{doi:10.1016/j.fusengdes.2011.12.018}
  \vspace{0.2cm}

\item D.G. Whyte \textit{et al.} \textit{Reactor similarity for
  plasma--material interactions in scaled-down tokamaks as the basis
  for the Vulcan conceptual design.} Fus. Eng. Design \textbf{87} (2012) 234.
  \href{http://dx.doi.org/10.1016/j.fusengdes.2011.12.011}{doi:10.1016/j.fusengdes.2011.12.011}\vspace{0.2cm}

\item Z.S. Hartwig and M. Zucchetti. \textit{Neutronics studies for a
  compact, high-field tokamak neutron source.}
  Fus. Sci. Tech. \textbf{60} (2011)
  725. \href{http://www.ans.org/pubs/journals/fst/a_12471}{Available online at http://www.ans.org/pubs/journals/fst/a\_12471}
\end{innerlist}

\vspace{0.5cm}

\textbf{Particle and radiation detector design, simulation, and data acquisition}
\begin{innerlist}
\item Z.S. Hartwig. \textit{The ADAQ framework: An integrated
    toolkit for data acquisition and analysis with real and simulated
    radiation detectors.} Nucl. Instr. and Meth. A \textit{In Press}, 2016.\vspace{0.2cm}
  
\item Z.S. Hartwig and P. Gumplinger. \textit{Simulating response functions
  and pulse shape discrimination for organic scintillation detectors
  with Geant4}. Nucl. Instr. and Meth. A \textbf{737} (2014) 155.\\
  \href{http://dx.doi.org/10.1016/j.nima.2013.11.027}{doi:10.1016/j.nima.2013.11.027}
  \vspace{0.2cm}

\item A. Inglis \textit{et al.} \textit{Glass panel Lithium-6 Detector.}
IEEE Conference on Homeland Security (2012).\\
\href{http://dx.doi.org/10.1109/THS.2012.6459887}{doi:10.1109/THS.2012.6459887}
\end{innerlist}

\vspace{0.5cm}

\textbf{Experimental particle physics (The Muon Lifetime Analysis (M$\mu$LAN) experiment)}
\begin{innerlist}
\item  D.M. Webber \textit{et al.} \textit{Measurement of the positive muon
  lifetime and determination of the Fermi constant to
  part-per-million precision.} Phys. Rev. Lett. \textbf{106} (2011)
  041803.
  \href{http://dx.doi.org/10.1103/PhysRevLett.106.041803}{doi:10.1103/PhysRevLett.106.041803}\vspace{0.2cm}

\item V. Tishchenko \textit{et al.} \textit{Detailed report of the
  MuLan measurement of the positive muon lifetime and determination of
  the Fermi constant.} Phys. Rev. D. \textbf{87} (2013)
    052003. 
    \href{http://dx.doi.org/10.1103/PhysRevD.87.052003}{doi:10.1103/PhysRevD.87.052003}
\end{innerlist}

\section{Reference Publications}
\textbf{A comprehensive physics and mathematics reference for magnetic fusion}
\begin{innerlist}
\item Z.S. Hartwig and Y.A. Podpaly. \textit{The Magnetic Fusion
  Energy Formulary}. Self-published, 2016.
  \href{http://www-internal.psfc.mit.edu/research/MFEFormulary}{Available
    online at http://www-internal.psfc.mit.edu/research/MFEFormulary}.
\end{innerlist}

\newpage
\makeheading{Zachary S. Hartwig, Ph.D.}

\section{Teaching Experience}
\begin{innerlist}
\item \textbf{UROP advisor}: Florent Sainct (2009), Jake Jurewicz (2011), Gabriel Ledoux (2012, 2013)
\item \textbf{Undergraduate thesis mentor}: Lauren Chilton, MIT Class of 2012.
\item \textbf{Teaching assistant}: 22.63: Engineering Principles for Fusion Reactors (Prof. D. Whyte). Spring 2012.
\item \textbf{Teaching assistant}: 22.105 : Electromagnetic Interactions (Prof. D. Whyte). Fall 2010.
\item \textbf{Private tutor}: High school physics for several Boston University Academy Students. 2005-2006. 
\end{innerlist}

\section{Leadership Experience}
\begin{innerlist}
\item \textbf{Organizer}: U.S. fusion student advocacy trip to 30 Congressional offices in Washington DC. June 2012.
\item \textbf{Mediator}: Conflict resolution, MIT Resistance for Easing Friction and Stress Program. January 2010.
\end{innerlist}

\section{Hardware Expertise}
\textbf{Detector Data Acquisition}
\begin{innerlist}
  \item CAEN S.p.A. data acquisition systems, Tektronix digital oscilloscopes
\end{innerlist}

\halfblankline

\textbf{Particle Detector Construction}
\begin{innerlist}
\item Scintillator crystals, photomultiplier tubes, silicon avalanche
  photodiodes, silicon photomultiplier, signal preamplifiers,
  microcontrollers, soldering, basic machining, vacuum hardware,
  detector test platforms
\end{innerlist}

\section{Computer expertise}

\textbf{Programming Languages}
\begin{innerlist}
  \item C, C$+$$+$, Python, IPython/IPython notebooks, Open MPI, Unix shell scripting, GNU make, Matlab, IDL, HTML
\end{innerlist}

\halfblankline

\textbf{Particle Transport and Nuclear Physics Codes}
\begin{innerlist}
  \item Geant4, MCNP6/5/X, DAGMC CAD-based neutronics, SRIM/TRIM, EASY, NJOY, TALYS, EMPIRE
\end{innerlist}

\halfblankline

\textbf{Data acquisition, storage and analysis}
\begin{innerlist}
  \item ROOT, MDSplus
  \item Lead developer of the ADAQ framework
\end{innerlist}

\halfblankline

\textbf{Computer-Aided Design (CAD) and Analysis}
\begin{innerlist}
  \item Solid Edge ST5, CUBIT Tool Suite, COMSOL Multiphysics
\end{innerlist}

\halfblankline

\textbf{Cloud Computing}
\begin{innerlist}
  \item Amazon Web Services (EC2 Cloud Compute Framework)
\end{innerlist}

\halfblankline

\textbf{Productivity Software}
\begin{innerlist}
  \item Windows OS, Linux OS (Fedora, RHEL, Ubuntu), Emacs,
    Subversion, Git, GitHub, \LaTeX, GIMP, Inkscape
\end{innerlist} 

% \section{Academic Coursework (MIT)}

% \newlength{\rcollengthtwo}\setlength{\rcollengthtwo}{3.1in}%

% \begin{tabular}[t]{@{}p{\textwidth-\rcollengthtwo-\spacewidth}@{}p{\spacewidth}@{}p{\rcollengthtwo}}%

% % Address box
% \parbox{\textwidth-\rcollengthtwo-\spacewidth}{%
%   \begin{innerlist}
%   \item   3.320 Atomistic Computer Materials Modeling
%   \item  18.085 Computational Science and Engineering
%   \item  22.071 Electronics, Signals, and Measurement
%   \item  22.101 Applied Nuclear Physics
%   \item  22.105 Electromagnetic Interactions
%   \item  22.106 Neutron Interactions and Applications
%   \item  22.51 Quantum Theory of Radiation Interactions
%   \item  22.611 Introduction to Plasmas Physics I
%   \end{innerlist}
% }

% & \parbox{\spacewidth}{\centering} &

% % Non-snail-mail contact information
% \parbox{\rcollengthtwo}{%

% \begin{innerlist}
% \item  22.612 Introduction to Plasmas Physics II
% \item  22.62 Fusion Energy
% \item  22.63 Engineering Principles of Fusion Reactors
% \item  22.70 Materials for Nuclear Applications
% \item  22.811 Sustainable Energy
% \item  22.814 Nuclear Non-Proliferation
% \item  22.93 Teaching Experience in Nuclear Science
% \end{innerlist}
% }

% \end{tabular}

\end{document}

%%%%%%%%%%%%%%%%%%%%%%%%%% End CV Document %%%%%%%%%%%%%%%%%%%%%%%%%%%%%

%----------------------------------------------------------------------%
% The following is copyright and licensing information for
% redistribution of this LaTeX source code; it also includes a liability
% statement. If this source code is not being redistributed to others,
% it may be omitted. It has no effect on the function of the above code.
%----------------------------------------------------------------------%
% Copyright (c) 2007, 2008, 2009, 2010, 2011 by Theodore P. Pavlic
%
% Unless otherwise expressly stated, this work is licensed under the
% Creative Commons Attribution-Noncommercial 3.0 United States License. To
% view a copy of this license, visit
% http://creativecommons.org/licenses/by-nc/3.0/us/ or send a letter to
% Creative Commons, 171 Second Street, Suite 300, San Francisco,
% California, 94105, USA.
%
% THE SOFTWARE IS PROVIDED "AS IS", WITHOUT WARRANTY OF ANY KIND, EXPRESS
% OR IMPLIED, INCLUDING BUT NOT LIMITED TO THE WARRANTIES OF
% MERCHANTABILITY, FITNESS FOR A PARTICULAR PURPOSE AND NONINFRINGEMENT.
% IN NO EVENT SHALL THE AUTHORS OR COPYRIGHT HOLDERS BE LIABLE FOR ANY
% CLAIM, DAMAGES OR OTHER LIABILITY, WHETHER IN AN ACTION OF CONTRACT,
% TORT OR OTHERWISE, ARISING FROM, OUT OF OR IN CONNECTION WITH THE
% SOFTWARE OR THE USE OR OTHER DEALINGS IN THE SOFTWARE.
%----------------------------------------------------------------------%
